\subsection{Объединённое множество решений}

\begin{definition}[ИСЛАУ]
	ИСЛАУ называется уравнение вида:
	\begin{equation}
		\mathbf{A}x = \mathbf{b},
	\end{equation}
	где $\mathbf{A} \in \mathbb{IR}^{m \times n}, \mathbf{b} \in \mathbb{IR}^m$, $x \in \mathbb{R}^n$ -- неизвестное.
\end{definition}

\begin{definition}[Объединённое множество решений]
	Объединённым множеством решений называется множество $\Xi_{uni}(\mathbf{A}, \mathbf{b})$, такое что существуют такие $A \in \mathbf{A}$ и $b \in \mathbf{b}$, для которых выполнено $Ax = b$:
	\begin{equation}
	\Xi_{uni}(\mathbf{A}, \mathbf{b}) = \{ x \in \mathbb{R}^n \; | \; \exists A \in \mathbf{A}, \; b \in \mathbf{b}: Ax = b \}
	\end{equation}
\end{definition}

Для нахождения объединённого множества решений в случае, когда одна размерность матрицы системы составляет 2 или 3, используется в сущности геометрический подход: для каждое уравнение системы задаёт линейно ограниченное множество. Пересечение этих множеств и задаёт объединённое множество решений. Пример можно найти в \cite[стр. 39, рис. 15]{intv}.

Программной реализацией являются функции \textit{EqnWeakR2, EqnWeakR3} пакетов \textit{IntLinIncR2, IntLinIncR3} (см. приложения).

Кроме того, для ИСЛАУ с квадратной матрицей существуют интервальные аналоги классических численных методов решения СЛАУ, которые позволяют оценить объединённое множество решений ИСЛАУ, опираясь на начальную оценку.

Интервальный метод Гаусса-Зейделя в точности повторяет классический метод Зейделя, сходится при любом начальном приближении, если $\| -(L + D) ^ {-1}U \| < 1$, где 
$L=
\begin{pmatrix}
0 & 0 & \dots & \dots & 0 \\
a_{21} & 0 & \dots & \dots & 0 \\
a_{31} & a_{32} & 0 & \dots & 0 \\
\vdots & \vdots & \vdots & \ddots & \vdots \\
a_{n1} & a_{n2} & a_{n3} & \dots & 0
\end{pmatrix}
$, 
$D=
\begin{pmatrix}
a_{11} & 0 & \dots & 0 \\
0 & a_{22} & \dots & 0 \\
\vdots & \vdots & \ddots & \vdots \\
0 & 0 & \dots & a_{nn}
\end{pmatrix}
$,
$R =
\begin{pmatrix}
0 & a_{12} & \dots & \dots & a_{1n} \\
0 & 0 & a_{23} & \dots & a_{2n} \\
0 & 0 & 0 & \dots & a_{3n} \\
\vdots & \vdots & \vdots & \ddots & \vdots \\
0 & \dots & \dots & \dots & 0
\end{pmatrix}
$

Метод Кравчика является разновидностью метода простых итераций:
\begin{equation}
\mathbf{x}^{(k+1)} = C\mathbf{x}^{(k)} + \mathbf{d}
\end{equation}

Согласно теореме Банаха о неподвижной точке такой процесс сходится, если $\|C\| < 1$. Поскольку выполнено: $\underset{Cv = \lambda_C v }{\textrm{max}} |\lambda_C| \leq \|C\|$, для удобства оценки сходимости можно использовать более сильное требование: $\rho(C) < 1$.

В методе Кравчика полагается:
\begin{equation}
C=(I - \Lambda \mathbf{A}), \; \Lambda = (\textrm{mid} \mathbf{A}) ^ {-1}
\end{equation}

\begin{equation}
d=\Lambda \mathbf{b}
\end{equation}

\subsection{Допусковое множество решений}

\begin{definition}[Допусковое множество решений]
	Допусковым множеством решений называется множество $\Xi_{tol}(\mathbf{A}, \mathbf{b})$, такое что для любой $A \in \mathbf{A}$ выполнено $Ax \in \mathbf{b}$:
	\begin{equation}
	\Xi_{tol}(\mathbf{A}, \mathbf{b}) = \{ x \in \mathbb{R}^n \; | \; \forall A \in \mathbf{A}: Ax \in \mathbf{b} \}
	\end{equation}
\end{definition}

Из-за того, что решение должно существовать для любой точечной матрицы ИСЛАУ, зачастую допусковое множество оказывается пустым там, где это совсем неочевидно на первый взгляд. Кроме того, в задачах, где простые эвристические решения, например, ``средней'' или ``крайней'' точечной СЛАУ не существуют, допусковое множество оказывается непустым. Эти примеры также можно найти в \cite{intv}.

Программной реализацией вычисления допускового множества являются функция \textit{[V, P1, P2, P3, P4] = EqnTol2D(infA, supA, infb, supb)} для двумерной системы и \textit{EqnTol3D} с аналогичными аргументами и списком возвращаемых значений для трёхмерной.

\subsubsection{Распознающий функционал}

Для полного исследования разрешимости задачи о допусковом множестве ИСЛАУ используется понятие \textit{распознающего функционала}:

\begin{equation}
\textrm{Tol}(x, \mathbf{A}, \mathbf{b}) = \underset{1 \leq i \leq m}{\textrm{min}} \bigg \lbrace \textrm{rad} \mathbf{b}_i - \bigg| \textrm{mid} \mathbf{b}_i - \sum_{j=1}^{n} \mathbf{a}_{ij}x_j \bigg|  \bigg \rbrace
\end{equation}

\begin{theorem}
	Пусть $x \in \mathbb{R}^n$. Тогда $x$ принадлежит допусковому множество тогда и только тогда, когда $\textrm{Tol}(x, \mathbf{A}, \mathbf{b}) \geq 0$.
\end{theorem}

Данная теорема сильно напоминает теоремы, позволяющие доказать оптимальность найденного решения задачи линейного программирования. Это неудивительно: задача о поиске допускового множества (\textit{линейная задача о допусках}) представляет собой задачу о решении системы линейных неравенств.

Программной реализацией вычисления распознающего функционала и аргумента, в котором достигается его неотрицательное значение, является функция \textit{[tolmax, argmax] = tolsolvty(supA, infA, supb, infb)}.

\subsection{Оценка вариабельности решения ИСЛАУ}

Вариабельность решения СЛАУ -- это свойство системы, демонстрирующее насколько относительная погрешность в свободном столбце СЛАУ влияет на относительную погрешность решения. Известно, что вариабельность решения зависит от числа обусловленности матрицы:

\begin{equation}
\frac{\|\delta x\|}{\|x\|} \leq \textrm{cond}(A) \frac{\|\delta b\|}{\|b\|}, \; \text{где}
\end{equation}

\begin{equation}
\textrm{cond}(A) = \|A\| \cdot \|A^{-1}\|
\end{equation}

Аналогами для ИСЛАУ являются показатели абсолютной вариабельности \textit{ive (interval variability of the estimate)} и относительной \textit{rve (relative variability of the estimate)}:

\begin{equation}
\textrm{ive}(\mathbf{A}, \mathbf{b}) = \sqrt{n} \cdot \bigg(\underset{A \in \mathbf{A}}{\textrm{min}} \; \textrm{cond} A \bigg) \cdot \| \textrm{arg max Tol}\| \cdot \frac{\max \textrm{Tol}}{\|\mathbf{b}\|}
\end{equation}

\begin{equation}
\textrm{rve}(\mathbf{A}, \mathbf{b}) = \bigg(\underset{A \in \mathbf{A}}{\textrm{min}} \; \textrm{cond} A \bigg) \cdot \max \textrm{Tol}	
\end{equation}