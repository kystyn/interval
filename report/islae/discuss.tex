В результате проделанной работы были найдены допусковые множества решений для обеих задач и оценки вариабельности. Из полученных результатов видно, что ive позволяет произвести качественную оценку линейного размера допускового множества решений: грубая оценка длины отрезка, соединяющего наиболее отдалённые точки в допусковом множестве первой задачи -- 0.40. 
Тот же параметр для второй задачи -- 0.80. Параметр ive, во-первых, имеет тот же порядок величины, а во-вторых, наблюдается, что при меньшем размахе допускового множества ive меньше, а значит, оценка вариабельности действительно позволяет качественно оценивать размер допускового множества решений.

Также видно, что оценка допускового множества квадратом с центром в точке максимума распознающего функционала и шириной ive покрывает значительную часть допускового множества, что позволяет оценить примерную его локализацию, но совершенно не берёт во внимание его форму.