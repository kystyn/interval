Исследователь поведение субдифференциального метода Ньютона при решении ИСЛАУ с матрицей:

\begin{equation}
    \mathbf{A}=
    \begin{pmatrix}
    [3, 4] & [5, 6] \\
    [-1, 1	] & [-3, 1] \\
    \end{pmatrix}
\end{equation}

и правой частью:

\begin{equation}
\mathbf{b}=
\begin{pmatrix}
[-3, 4] \\
[-1, 2]
\end{pmatrix}
\end{equation}

Объяснить, по каким причинам метод расходится при любом допустимом релаксационном параметре $\tau \in [0; 1]$.