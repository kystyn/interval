\subsection{Субдифференциальный метод Ньютона}

Кратко напомним суть субдифференциального метода Ньютона.Итерация выглядит следующим образом:

\begin{equation*}
	x^{(k)} \leftarrow x^{(k - 1)} - \tau(D^{(k - 1)})^{-1}\mathcal{F}(x^{(k - 1)})
\end{equation*}

где $D^{(k - 1)}\mathcal{F}(x^{(k - 1)})$ -- какой-нибудь субградиент в $x^{(k - 1)}$, $\tau \in [0; 1]$ -- постоянный коэффициент. Алгоритм заканчивает работу, когда $\| \mathcal{F}(x^{(k)}) \| < \varepsilon$.

В качестве такой начальной точки обычно принимают решение точечной СЛАУ:
\begin{equation}
(\textrm{mid} \mathbf{C})^{\tilde{}}x = \texttt{sti}(\mathbf{d})
\end{equation}

 (\cite[стр. 607]{shary})

\subsection{Вычисление субградиента}

Для вычисления субградиента используются формулы:

\begin{align}
\partial\mathcal{F}_i=-\sum_{j=1}^{n}\partial(\underline{\mathbf{c}_{ij}[-x_j, x_{j+n}]}), \;i = \overline{1,n} \\
\partial\mathcal{F}_i=\sum_{j=1}^{n}\partial(\overline{\mathbf{c}_{ij}[-x_j, x_{j+n}]}), \;i = \overline{n+1,2n}
\end{align}

Более детальное описание можно найти в \cite{shary} на стр. 610-613.

\subsection{Приближённое решение ИСЛАУ с точечной прямоугольной матрицей }

\subsubsection{Первый вариант}

В случае переопределённой матрицы самым простым способом решения задачи будет произвольный выбор строк матрицы ИСЛАУ и соответствующих компонент правого столбца с проверкой соблюдения условий сходимости (неособенности модуля выбранной матрицы).

Можно произвести этот процесс несколько раз. В таком случае задействуется большее количество уравнений из системы, так, чтобы по его окончании в решении оказались задействованы все условия. Тогда итоговое решение можно представить как пересечение всех найденных.

Недостатки данного подхода следующие:
\begin{enumerate}
	\item Он не детерминирован. Однажды получив удачную конфигурацию матриц, можно более её не получить, если, конечно, не логгировать данные такого рода.
	
	\item Эту конфигурацию в принципе получить тяжело: исключая из матрицы высоты 256 36 линейно независимых строк, можно очень долго произвольным образом выискивать следующую группу линейно независимы строк. И с огромной вероятностью в конце концов таких групп просто не останется.
\end{enumerate}

Поэтому был применён упрощённый подход: на каждой итерации выбиралась произвольная абсолютно неособенная матрица и соответствующий правый столбец. При этом не вводилось требование задействовать все возможные строки.

\subsubsection{Второй вариант}

Другой подход -- поэтапное исключение строк, не понижающих ранг рассматриваемой матрицы: на каждой итерации делается попытка удалить одну строку. Если удаление не понизило ранг, то оно делается окончательно, и процесс переходит на следующую строку. Если же удаление строки ведёт к понижению ранга матрицы, то эту строку мы не трогаем и движемся к следующей строке. Такой подход имеет следующие преимущества:
\begin{enumerate}
	\item Детерминированность. Каждый запуск программы мы будем получать одно и то же решение.
	
	\item Необходимо решать ИСЛАУ только один раз.
\end{enumerate}

В итоге по причине детерминированности было принято решение использовать именно второй подход.