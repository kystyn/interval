\subsection{Субдифференциальный метод Ньютона}

Кратко напомним суть субдифференциального метода Ньютона.Итерация выглядит следующим образом:

\begin{equation*}
	x^{(k)} \leftarrow x^{(k - 1)} - \tau(D^{(k - 1)})^{-1}\mathcal{F}(x^{(k - 1)})
\end{equation*}

где $D^{(k - 1)}\mathcal{F}(x^{(k - 1)})$ -- какой-нибудь субградиент в $x^{(k - 1)}$, $\tau \in [0; 1]$ -- постоянный коэффициент. Алгоритм заканчивает работу, когда $\| \mathcal{F}(x^{(k)}) \| < \varepsilon$.

В качестве такой начальной точки обычно принимают решение точечной СЛАУ:
\begin{equation}
(\textrm{mid} \mathbf{C})^{\tilde{}}x = \texttt{sti}(\mathbf{d})
\end{equation}

 (\cite[стр. 607]{shary})

\subsection{Вычисление субградиента}

Для вычисления субградиента используются формулы:

\begin{align}
\partial\mathcal{F}_i=-\sum_{j=1}^{n}\partial(\underline{\mathbf{c}_{ij}[-x_j, x_{j+n}]}), \;i = \overline{1,n} \\
\partial\mathcal{F}_i=\sum_{j=1}^{n}\partial(\overline{\mathbf{c}_{ij}[-x_j, x_{j+n}]}), \;i = \overline{n+1,2n}
\end{align}

Более детальное описание можно найти в \cite{shary} на стр. 610-613.

\subsection{Приближённое решение ИСЛАУ с точечной прямоугольной матрицей}

В случае переопределённой матрицы самым простым способом решения задачи будет произвольный выбор строк матрицы ИСЛАУ и соответствующих компонент правого столбца с проверкой соблюдения условий сходимости (неособенности модуля выбранной матрицы).

Можно произвести этот процесс несколько раз, так, чтобы по его окончании в решении оказались задействованы все условия. Тогда итоговое решение можно представить как пересечение всех найденных.