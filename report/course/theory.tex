\subsection{Коррекция СЛАУ. Метод Лаврентьева}

Для регуляризации ИСЛАУ с интервальной матрицей был использован метод Лаврентьева.

Суть метода в следующем: матрица ИСЛАУ $\mathbf{A}$ заменяется на некоторую ``расширенную'' матрицу 
\begin{equation}
\mathbf{A} + \theta \mathbf{E}
\end{equation}

где $\mathbf{E}$ -- матрица, состоящая из интервалов $[-1, 1]$.

\subsection{Эвристические методы коррекции нульсодержащих интервалов}

Было выдвинуто предположение, что проблема неразрешимости имеет место быть из-за нульсодержащих интервалов. Соответственно, суть данной эвристики состоит в том, чтобы посмотреть, как метод ведёт себя при коррекции интервалов таким образом, чтобы в него не попадал ноль. Когда соответствующий интервал (или группа интервалов) обнаруживается, с помощью метода половинного деления производится уточнение интервала.

Метод может применяться как к матрице, так и к столбцу свободных членов.
