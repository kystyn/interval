В результате регуляризации системы методом Лаврентьева был получен ожидаемый результат: регуляризация провалилась. Действительно, сколько ни расширяй интервал, он всё так же будет нульсодержащим и система останется неразрешимой.

Эвристический метод, применённый к матрице, подтвердил эту гипотезу: действительно, если  скорректировать один из интервалов так, чтобы ноль был его границей, то ИСЛАУ становится разрешимой. Было проверено, что при подстановке сколь угодно малого положительного числа вместо нуля ИСЛАУ снова становится неразрешимой (слова ``сколь угодно малый'', конечно же, стоит воспринимать в контексте компьютерных вычислений).

Достаточно неожиданный результат был получен при коррекции столбца свободных членов. Было обнаружено, что ИСЛАУ становится разрешимой при весьма незначительной коррекции одного из интервалов. При этом интервал не перестал быть нульсодержащим.

Из этого можно сделать следующий вывод: знакопеременность концов интервалов матрицы ИСЛАУ не является критическим обстоятельством для работы метода, однако в такой ситуации очень многое зависит от выбора начального приближения -- ведь именно оно (и только оно!) определяется свободным столбцом ИСЛАУ.