\begin{itemize}
	\item В первую очередь был поставлен вычислительный эксперимент, цель которого -- показать, что, по крайней мере, на значительном множестве допустимых релаксационных параметров метод расходится. Для этого метод был запущен с $\tau$, принадлежащим равномерной сетке от 0 до 1 с шагом 0.05.
	
	\item В результате применения метода Лаврентьева в совокупности с бинарным поиском было обнаружено, что ни при каких $\theta$ система не становится разрешимой.
	
	\item В результате применения эвристического метода, описанного в теоретическом разделе, было обнаружено два способа коррекции системы:
	
	\begin{enumerate}
		\item 
		\begin{equation}
		\mathbf{A}=
		\begin{pmatrix}
		[3, 4] & [5, 6] \\
		[-1, 1] & [-3, 0]
		\end{pmatrix}
		\end{equation}
		
		Свободный столбец остаётся при этом неизменным.
		
		Найдено формальное решение:
		
		\begin{equation}
		\mathbf{x}=
		\begin{pmatrix}
		[-0.250, 1.000]\\
		[-0.333, 0.000]
		\end{pmatrix}
		\end{equation}
		
		\item
		\begin{equation}
		\mathbf{b}=
		\begin{pmatrix}
		[-3, 3.002]\\
		[-1, 2]
		\end{pmatrix}
		\end{equation}
		
		Матрица остаётся при этом неизменной.
		
		Найдено формальное решение:
		
		\begin{equation}
		\mathbf{x}=
		\begin{pmatrix}
		[-0.000, 0.500]\\
		[-0.500, 0.166]
		\end{pmatrix}
		\end{equation}
	\end{enumerate}
	
\end{itemize}