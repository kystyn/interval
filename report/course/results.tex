\begin{itemize}
	\item В первую очередь был поставлен вычислительный эксперимент, цель которого -- показать, что, по крайней мере, на значительном множестве допустимых релаксационных параметров метод расходится. Для этого метод был запущен с $\tau$, принадлежащим равномерной сетке от 0 до 1 с шагом 0.05.
	
	\item В результате применения метода Лаврентьева в совокупности с бинарным поиском был найден наименьший коэффициент растяжения интервала -2.14 с точностью до сотых.
	
	Итоговая матрица системы имеет вид:
	
	\begin{equation}
	\mathbf{A}=
	\begin{pmatrix}
	[0.85, 1.85] & [2.85, 3.85] \\
	[-3.14, -1.14] & [-5.14, -1.14] \\
	\end{pmatrix}
	\end{equation}
	
\end{itemize}