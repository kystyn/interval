\subsection{Концепция решения интервальных уравнений}

Речь идёт о решении интервальных уравнений вида $\mathbf{C}x = \mathbf{d}$ в полной интервальной арифметике.

Данная задача могла бы быть решена классическими численными методами, если бы не одно обстоятельство: даже полная интервальная арифметика Каухера не является линейным пространством. Ввиду данного обстоятельства предлагается следующий подход: необходимо сформировать взаимно-однозначное отображение из арифметики Каухера в некоторое линейное пространство. Поскольку речь идёт о конечномерном случае, в качестве линейного пространства можно рассматривать пространство вещественных чисел, поскольку все конечномерные линейные пространства одной размерности изометрически изоморфны.
Далее задача решается в линейном пространстве ранее изученными методами, а затем полученный результат обратно отображается в арифметику Каухера.

Ясно, что $n$-мерное пространство Каухера $\mathbb{KR}^n$ должно отображаться в $2n$-мерное евклидово пространство $\mathbb{R}^{2n}$ ввиду природы элементов арифметики Каухера (интервалов).

Далее речь пойдёт о конкретных отображениях и сопутствующей терминологии.

\subsection{Погружение}

Пусть $\varphi$ -- некоторое отображение на интервальном пространстве, задействованное при решении задачи:
\begin{equation}
\varphi: \mathbb{KR}^n \rightarrow \mathbb{KR}^n
\end{equation}

Пусть имеется отображение:
\begin{equation}
i: \mathbb{KR}^n \rightarrow \mathbb{R}^{2n}
\end{equation}

Тогда образ данной задачи в линейном пространстве имеет вид:

\begin{equation}
(i \circ \varphi \circ i^{-1})(y), \; y \in \mathbb{R}^{2n}
\end{equation}

Поскольку отображение $i$ должно быть обратимым, оно обязано быть биективным.

\begin{definition}
	Биективное отображение, действующее из $\mathbb{KR}^{n}$ в $\mathbb{R}^{2n}$, называется \textbf{погружением} в линейное пространство.
\end{definition}

\begin{definition}
\textbf{Стандартным погружением} называется отображение
\begin{equation}
\mathtt{sti}(\textit{\textbf{x}}): 
\begin{pmatrix}
\textit{\textbf{x}}_1 & \dots & \textit{\textbf{x}}_n
\end{pmatrix}
\rightarrow
\begin{pmatrix}
-\underline{\textit{\textbf{x}}_1} & \dots & -\underline{\textit{\textbf{x}}_n} & 
\overline{\textit{\textbf{x}}_1} & \dots &
\overline{\textit{\textbf{x}}_n}
\end{pmatrix}
\end{equation}
\end{definition}

Оно действительно является инъективным и сюръективным, что легко проверяется по определению этих понятий \cite[Лекция 11]{intv}.

Кроме того, данное отображение индуцирует частичный порядок на $\mathbb{R}^{2n}$, что необходимо для корректного определения понятия выпуклости, используемого при построении численных методов оптимизации и решения ИСЛАУ \cite[стр. 580]{shary}.

Рассмотрим случай, когда оператор $\varphi$ задаётся квадратной точечной матрицей $Q$:

\begin{equation}
\varphi(x)=Qx
\end{equation}

В таком случае оператор $\mathtt{sti} \circ \varphi \circ \mathtt{sti}^{-1}$ будет являться линейным, определённым \textit{знаково-блочной матрицей}:

\begin{equation}
Q\tilde{} =
\begin{pmatrix}
Q^+ & \vrule & Q^- \\
\hline
Q^- & \vrule & Q^+
\end{pmatrix}
\end{equation}

Покажем это.

\begin{lemma} \label{l1} \cite[стр. 583]{shary}
	Имеет место следующее свойство умножения интервала на число:
	\begin{equation}
	\begin{cases}
	\underline{q \cdot \textbf{x}} = q^+\underline{\textbf{x}} - q^-\underline{\textbf{x}} \\
	
	\overline{q \cdot \textbf{x}} = -q^-\overline{\textbf{x}} + q^+\overline{\textbf{x}}
	\end{cases}
	\end{equation}
	где
	\begin{equation}
	q^+ =
	\begin{cases}
	q & q \geq 0 \\
	0 & q < 0
	\end{cases}
	\end{equation}
	
	\begin{equation}
	q^- =
	\begin{cases}
	-q & q \leq 0 \\
	0 & q > 0
	\end{cases}
	\end{equation}
	
	Доказательство -- по определению умножения числа на интервал в арифметике Каухера.
	$\blacksquare$
\end{lemma}

Пусть $y \in \mathbb{R}^{2n}$.
Тогда:
\begin{equation}
\mathtt{sti}^{-1}(y)=
\begin{pmatrix}
[-y_1, y_{n + 1}] \dots [-y_n, y_{2n}]
\end{pmatrix}
\end{equation}

\begin{equation}
(\varphi \circ \mathtt{sti}^{-1})(y)=
\begin{pmatrix}
q_{11}[-y_1, y_{n + 1}] + \dots + q_{1n} [-y_n, y_{2n}] \\
\vdots \\
q_{n1}[-y_1, y_{n + 1}] + \dots + q_{nn} [-y_n, y_{2n}]
\end{pmatrix}
\end{equation}

Далее по лемме \ref{l1}:
\begin{equation}
(\mathtt{sti} \circ \varphi \circ \mathtt{sti}^{-1})(y)=
\begin{pmatrix}
q_{11}^+y_1 + \dots + q_{1n}^+ y_n - q_{11}^-y_{n+1} - \dots - q_{1n}^- y_{2n}] \\
\vdots \\
q_{n1}^+y_1 + \dots + q_{nn}^+ y_n - q_{n1}^-y_{n+1} - \dots - q_{nn}^- y_{2n}]
\end{pmatrix}
=
Q\tilde{ }y
\end{equation}

Кроме того, имеет место теорема \cite[Лекция 11]{intv}:

\begin{theorem}
	Для точечной матрицы $Q \in \mathbb{R}^{n \times n}$ следующие условия равносильны:
	\begin{itemize}
		\item $Q\mathbf{x} = 0$ в интервальном пространстве $\mathbb{KR}^{n}$ тогда и только тогда, когда $\mathbf{x} = 0$
		
		\item Матрица $Q\tilde{} \in \mathbb{R}^{2n \times 2n}$, знаково-блочная для $Q$, является неособенной
		
		\item Неособенной является $|Q|$
	\end{itemize}
\end{theorem}

\begin{corollary}
	Оператор $\varphi$, введённый ранее, необратим тогда и только тогда, когда его матрица является абсолютно неособенной.
\end{corollary}

Таким образом, теперь мы можем перейти от решения ИСЛАУ $A\mathbf{x}=\mathbf{b}$ с абсолютно неособенной матрицей $A$ к решению стандартной СЛАУ в линейном пространстве. 

\subsection{Решение ИСЛАУ с точечной матрицей с помощью погружений}
Получим явный вид оператора, обратного к $\varphi$.

Пусть $y \in \mathbb{R}^{2n}$. Тогда:

\begin{align}
(\mathtt{sti} \circ \varphi \circ \mathtt{sti}^{-1})(y) = Q\tilde{}y \Leftrightarrow \\
Q\tilde{}^{-1}y = (\mathtt{sti} ^ {-1} \circ \varphi^{-1} \circ \mathtt{sti})(y) \Leftrightarrow \\
(\mathtt{sti} ^ {-1} \circ Q\tilde{}^{-1})y = (\varphi^{-1} \circ \mathtt{sti})(y) \Leftrightarrow \\
[\mathbf{x} = \mathtt{sti}^{-1}(y), \; x \in \mathbb{KR}^n] \Leftrightarrow \\
\varphi^{-1}(\mathbf{x}) = (\mathtt{sti} ^ {-1}( Q\tilde{}^{-1}(\mathtt{sti}(\mathbf{x}))))
\end{align}

Итак, перезапишем ИСЛАУ $Q\mathbf{x} = b$ её в виде: $\varphi(\mathbf{x})=b$. Тогда $\mathbf{x} = \varphi^{-1}(b)$. Окончательный ответ:

\begin{equation} \label{ans_x}
\mathbf{x} = (\mathtt{sti} ^ {-1}( Q\tilde{}^{-1}(\mathtt{sti}(\mathbf{b})))
\end{equation}

Таким образом, мы свели решение интервальной СЛАУ к поиску обратной точечной матрицы. Однако зачастую легче решать не задачу поиска обратной матрицы, а задачу поиска решения СЛАУ: для этой задачи существует множество точных (LDR-, LU-разложение, методы вращений и отражений) и численных методов (методы простых итераций, Якоби, Зейделя). В таком случае \ref{ans_x} лучше представить как совокупность двух задач:

\begin{align} \label{ans_b}
Q\tilde{} z = \mathtt{sti}(\mathbf{b}), \; z \in \mathbb{R}^{2n} \\
\mathbf{x} = (\mathtt{sti} ^ {-1}(z))
\end{align}

\subsection{Частичный порядок в $\mathbb{KR}^n$. Порядковая выпуклость. Условия разрешимости ИСЛАУ}

Рассмотрим следующий подход к решению ИСЛАУ.
Пусть сформулирована задача: $\mathbf{Cx}=\mathbf{d}$.
Рассмотрим оператор:

\begin{equation}
\mathcal{F}(y)=\mathtt{sti}(\mathbf{C}\mathtt{sti}^{-1}(y) \ominus \mathtt{d}): \mathbb{R}^{2n} \rightarrow \mathbb{R}^{2n}
\end{equation}

Приравнивая $\mathcal{F}$ к нулю мы получаем задачу, индуцированную погружением исходной задачи в линейное пространство.

Хорошо развиты методы решения задача поиска корня выпуклого функционала. Однако предложенная постановка задачи составлена именно относительно отображения, действующего на конечномерном пространстве высокой размерности.

Рассмотренные ранее подходы, применимые к функционалам, как выясняется, можно применить и в этой ситуации, однако для этого необходимо перенести сопутствующую терминологию на многомерный случай.

\begin{definition}
	Пусть $F: \mathbb{R}^p \rightarrow \mathbb{R}^q$.
	
	Пусть также в $\mathbb{R}^q$ задан частичный порядок $\preccurlyeq$. Тогда отображение $F$ называется \textbf{порядково выпуклым относительно $\preccurlyeq$}, если
	\begin{equation*}
	F(\lambda u + (1 - \lambda) v) \preccurlyeq \lambda F(u) + (1 - \lambda) F(v) \; \forall u, v \in \mathbb{R}^p \; \forall \lambda \in [0; 1]
	\end{equation*}.
\end{definition}

Свойство (порядковой) выпуклости необходимо для построения ньютоновских методов. Однако получить его можно далеко не во всех случаях. Кроме того, необходимо описать механизм, задающий частичный порядок.

Введём для этого следующее понятие:

\begin{definition}
	Будем называть квадратную интервальную матрицу размера $n$ \textbf{построчно однородной}, если в каждой её строке все элементы являются либо только правильными интервалами, либо только неправильными.
\end{definition}

Частичный порядок определяется построчно однородной матрицей $\Subset$ следующим образом: пусть $\mathcal{I}'$ -- номера строк, в которых все интервалы являются правильными, а $\mathcal{I}''$ -- где неправильными. Тогда $\mathbf{u} \Subset \mathbf{v}, \; \mathbf{u, v} \in \mathbb{KR}^n$, если и только если:

\begin{equation}
\begin{cases}
\mathbf{u}_i \subseteq \mathbf{v}_i, & i \in \mathcal{I}' \\

\mathbf{v}_i \subseteq \mathbf{u}_i, & i \in \mathcal{I}''

\end{cases} 
\end{equation}

Стандартное погружение также индуцирует отношение порядка $\ll$ на евклидовом пространстве по отношению $\Subset$.

Теперь мы обладаем необходимой терминологией и способом совместного упорядочивания интервального и евклидова пространства.

Сформулируем условия, при которых отображение $\mathcal{F}$ будет порядково выпуклым.

\begin{theorem}
	Если матрица $\mathbf{C}$ обладает условием построчной однородности, то индуцированное отображение $\mathcal{F}$ является порядково выпуклым относительно порядка $\ll$.
\end{theorem}

\begin{definition}
	Пусть на $\mathbb{R}^q$ задан частичный порядок $\preccurlyeq$. Пусть   $F: \mathbb{R}^p \rightarrow \mathbb{R}^q$. Тогда его \textbf{субдифференциалом} называется такое множество линейных операторов $D: \mathbb{R}^p \rightarrow \mathbb{R}^q$, что:
	
	\begin{equation}
	D(v) \preccurlyeq F(x + v) - F(x) \; \forall v \in \mathbb{R}^p
	\end{equation}
	
	Элементы множества $D$ называются \textbf{субградиентом} и обозначаются $\partial_{\ll}F(x)$.
\end{definition}

\subsection{Субдифференциальный метод Ньютона}

Данный метод является модификацией метода градиентного спуска, который, однако, не накладывает на отображение требования дифференцируемости, а требует лишь порядковой выпуклости.

\begin{equation*}
	x^{(k)} \leftarrow x^{(k - 1)} - \tau(D^{(k - 1)})^{-1}\mathcal{F}(x^{(k - 1)})
\end{equation*}

где $D^{(k - 1)}\mathcal{F}(x^{(k - 1)})$ -- какой-нибудь субградиент в $x^{(k - 1)}$, $\tau \in [0; 1]$ -- постоянный коэффициент. Алгоритм заканчивает работу, когда $\| \mathcal{F}(x^{(k)}) \| < \varepsilon$.

В \cite{shary} утверждается, что при единичном $\tau$ метод даёт точный результат для полиэдральных функций, коей является $\mathcal{F}$, однако вариация этого коэффициента может помочь ускорить сходимость для других функций.

Для сходимости метода критично, чтобы $\mathcal{F}(x) \geq 0$. Если обеспечить выполнение этого условия на начальной итерации, то метод сойдётся. В качестве такой начальной точки подойдёт решение точечной СЛАУ:
\begin{equation}
(\textrm{mid} \mathbf{C})^{\tilde{}}x = \texttt{sti}(\mathbf{d})
\end{equation}

 (\cite[стр. 607]{shary})

\subsection{Вычисление субградиента}

Для вычисления субградиента используются формулы:

\begin{align}
\partial\mathcal{F}_i=-\sum_{j=1}^{n}\partial(\underline{\mathbf{c}_{ij}[-x_j, x_{j+n}]}), \;i = \overline{1,n} \\
\partial\mathcal{F}_i=\sum_{j=1}^{n}\partial(\overline{\mathbf{c}_{ij}[-x_j, x_{j+n}]}), \;i = \overline{n+1,2n}
\end{align}

Более детальное описание можно найти в \cite{shary} на стр. 610-613.

\subsection{Приближённое решение ИСЛАУ с точечной прямоугольной матрицей}

В случае переопределённой матрицы самым простым способом решения задачи будет произвольный выбор строк матрицы ИСЛАУ и соответствующих компонент правого столбца с проверкой соблюдения условий сходимости (неособенности модуля выбранной матрицы).

Можно произвести этот процесс несколько раз, так, чтобы по его окончании в решении оказались задействованы все условия. Тогда итоговое решение можно представить как пересечение всех найденных.