\subsection{Задача 1}

Субдифференциальный метод Ньютона вернул следующий результат:
\begin{equation}
\mathbf{x} = 
\begin{pmatrix}
1.000 \\
[1.444, 0.636] \\
[0.667, 1.273]
\end{pmatrix}
\end{equation}

Решение производилось с точностью до третьего знака. Невязка решения равна нулю (найденное решение является точным).

\begin{remark}
	Подразумевается ненулевая невязка при подстановке решения вплоть до последнего полученного знака. Однако предъявляется решение именно вплоть до третьего знака, поскольку запрос в методе был именно таковым. Однако, как мы знаем из теоретического раздела, данный метод отыскивает точное решение полиэдральных функций, поэтому полученный результат является отнюдь не удивительным, а закономерным.
\end{remark}

\subsection{Задача 2}

В качестве начального вектора был взят вектор, приложенный в письме (см. \ref{app}).

Сопутствующие задачи с квадратными матрицами были решены с помощью метода, использующего знаково-блочные матрицы. В результате был получен следующий вектор:
\begin{equation}
\mathbf{x}=
\begin{pmatrix}
[0.620, 0.379] \\
[0.706, 0.783] \\
[1.045, 0.952] \\
[0.941, 1.067] \\
[0.972, 0.521] \\
[0.377, 0.624] \\
[0.430, 0.067] \\
[0.646, 0.354] \\
[0.818, 0.691] \\
[1.247, 0.253] \\
[0.462, 0.542] \\
[0.228, 0.272] \\
[0.287, -0.275] \\
[1.596, -1.096] \\ 
[0.737, 0.261] \\
[5.643, -4.642] \\
[0.693, -0.195] \\
[0.376, -0.388] \\
[-0.036, 0.038] \\
[0.558, -0.062] \\
[1.467, -0.467] \\
[1.927, -0.952] \\
[0.660, -0.160] \\
[0.242, -0.237] \\
[0.374, 0.126] \\
[0.516, 0.485] \\
[0.733, 0.766] \\
[0.781, 0.712] \\
[0.515, 0.485] \\
[0.494, 0.006] \\ 
[0.473, 0.526] \\
[1.056, 0.443] \\
[1.172, 0.827] \\
[0.928, 1.072] \\ 
[0.794, 0.704] \\
[0.388, 0.611]
\end{pmatrix}
\end{equation}

Найденное решение удовлетворяет 144 уравнениям из 256.

Начальный вектор $x^{(0)}$ пересекается с итоговым результатом по 17 компонентам из 36.