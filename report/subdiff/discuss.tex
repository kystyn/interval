В результате решения обеих задач были найдены формальные решения ИСЛАУ. Была экспериментально проверена сходимость субдифференциального метода Ньютона к точному решению в случае полиэдральной функции. Также была проверена путём подстановки корректность найденного формального решения второй задачи. Представленное решение не является корректным для всех компонент, поскольку, во-первых, ИСЛАУ значительно переопределена и точного решения невозможно получить в принципе, а, во-вторых, благодаря высокому числу обусловленности построенной квадратной матрицы: его порядок составляет $10^{11}$, что также негативно сказывается на вычислительной точности.

Из приведённого графика видно, что форма найденного решения в целом повторяет форму центра вектора интервалов свободного столбца, что позволяет судить о состоятельности представленного эвристического метода.

Кроме того видно, что существует корреляция между суммой элементов в матрице и точностью найденного решения. В большинстве своём найденное решение совпало с серединой модельного, когда сумма элементов была большой. В случае де, когда сумма элементов меньше единицы, решение гарантированно оказывалось неудачным.