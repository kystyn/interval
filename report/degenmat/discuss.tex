Последовательное решение первой и второй задач наглядно демонстрирует область применения различных критериев. В задачах сравнительно высокой размерности критерий Баумана применять практически нерационально ввиду сверхэкспоненциального роста алгоритмической сложности задачи ($|vert \mathbf{A}| = 2 ^ {n ^ 2}$). В то же время, признак Бекка косвенно является приближённым ввиду численного вычисления обратной матрицы и матричного спектра, что является недостатком метода.
Из приведённого графика \ref{pic:degenmat} видно, что при росте размерности матрицы начало луча, при попадании $\varepsilon$ в который матрица становится особенной, становится всё ближе к нулю, то есть получить вырожденную матрицу становится проще.