\begin{definition}
	Интервальная матрица $\mathbf{A} \in \mathbb{IR}^{n \times n}$ называется \textbf{неособенной}, 
	если неособенны все точечные матрицы $A \in \mathbf{A}$. 
\end{definition}

\begin{definition}
	Если же все точечные матрицы являются особенными, то  $\mathbf{A}$ называется \textbf{особенной}.
\end{definition}

\begin{theorem}
	\textbf{Критерий Баумана} \cite{bazhenov}. Интервальная матрица неособенна тогда и только тогда, когда определители всех её крайних матриц имеют одинаковый знак.
\end{theorem}

%\begin{definition}
%	Говорят, что интервальная матрица имеет \textbf{диагональное преобладание}, если все точечные матрицы, содержащиеся в ней, имеют диагональное преобладание.
%\end{definition}

\begin{theorem}
	\textbf{Признак Бекка} \cite{bazhenov}. Пусть $mid \mathbf{A}$ неособенна ($\mathbf{A} \in \mathbb{IR}^{n \times n}$) и
	$$\rho(\mid (mid \mathbf{A})^{-1} \mid \cdot rad \mathbf{A}) < 1$$
	Тогда $\mathbf{A}$ неособенна.
\end{theorem}
