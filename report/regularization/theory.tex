\subsection{Объединённое множество решений}

Напомним, распознающий функционал имеет вид:

\begin{equation}
\textrm{Tol}(x, \mathbf{A}, \mathbf{b}) = \underset{1 \leq i \leq m}{\textrm{min}} \bigg \lbrace \textrm{rad} \mathbf{b}_i - \bigg| \textrm{mid} \mathbf{b}_i - \sum_{j=1}^{n} \mathbf{a}_{ij}x_j \bigg|  \bigg \rbrace
\end{equation}

Имеет место следующая теорема:

\begin{theorem}
	Пусть $x \in \mathbb{R}^n$. Тогда $x$ принадлежит допусковому множество тогда и только тогда, когда $\textrm{Tol}(x, \mathbf{A}, \mathbf{b}) \geq 0$.
\end{theorem}

Нетрудно заметить, что в случае неразрешимости интервальной системы одним из простейших способов коррекции системы будет расширение интервалов свободного столбца: повышая радиус интервала, мы увеличиваем значение распознающего функционала. Таким образом, можно увеличить интервал настолько, чтобы обеспечить неотрицательность функционала.

\begin{equation*}
\mathbf{b}=
\begin{pmatrix}
[\textrm{mid b}_1 - \textrm{rad b}_1, \textrm{mid b}_1 + \textrm{rad b}_1] \\
[\textrm{mid b}_2 - \textrm{rad b}_2, \textrm{mid b}_2 + \textrm{rad b}_2] \\
[\textrm{mid b}_3 - \textrm{rad b}_3, \textrm{mid b}_3 + \textrm{rad b}_3]
\end{pmatrix}
\rightarrow
\end{equation*}
\begin{equation*}
\tilde{\mathbf{b}}=
\begin{pmatrix}
[\textrm{mid b}_1 - w_1 \textrm{rad b}_1, \textrm{mid b}_1 + w_1 \textrm{rad b}_1] \\
[\textrm{mid b}_2 - w_2 \textrm{rad b}_2, \textrm{mid b}_2 + w_2 \textrm{rad b}_2] \\
[\textrm{mid b}_3 - w_3 \textrm{rad b}_3, \textrm{mid b}_3 + w_3 \textrm{rad b}_3]
\end{pmatrix}
\end{equation*}

При этом преследуется следующая цель: масштабирование интервала должно быть минимальным в определённом смысле.

\subsection{Интервальная регуляризация}

В этом разделе речь пойдёт о методе, который позволяет провести корректировку ИСЛАУ посредством решения задачи линейного программирования -- $l_1$-\textit{регуляризации} \cite{intv}.

Минимизируется первая норма вектора 
$w = 
\begin{pmatrix}
w_1 & w_2 & w_3
\end{pmatrix}^T \in \mathbb{R}_+^3$: $\|w\|_1 \rightarrow \min$.

Ясно, что при таком подходе нельзя строить задачу оптимизации с непосредственным использованием распознающего функционала ввиду его нелинейности. 

Будем опираться на определение допускового множества решений напрямую: требуется, чтобы найденный вектор $x$ при действии на него оператором $A$ попал в интервал $\tilde{\mathbf{b}}$.

Итак, требуется:
\begin{equation}
\begin{cases}
	\exists x \in \mathbb{R}^3: A \cdot x \in \tilde{\mathbf{b}} \\
	\|w\|_1 = w_1 + w_2 + w_3 \rightarrow \min
\end{cases}
\end{equation}

Введём 
$u=
\begin{pmatrix}
x \\ w
\end{pmatrix}$.
Тогда общий вид задачи ЛП перезаписывается следующим образом: 

\begin{equation}
\begin{cases}
\begin{pmatrix}
0 & 0 & 1 & 1 & 1
\end{pmatrix} \cdot u \; (== \|w\|_1) \rightarrow \underset{u}{\min} \\
u_{4,5,6} \geq 0 \\
\begin{pmatrix}
-A & -\textrm{diag}(\textrm{rad} \mathbf{b}) \\
 A & -\textrm{diag}(\textrm{rad} \mathbf{b})
\end{pmatrix}
\cdot
u
\leq 
\begin{pmatrix}
-\textrm{mid} \mathbf{b} \\
\textrm{mid} \mathbf{b}
\end{pmatrix}
\end{cases}
\end{equation}

\begin{remark}
	Последнее уравнение эквивалентно:
	\begin{equation*}
	\begin{cases}
		-Ax - \textrm{diag}(\textrm{rad} \mathbf{b}) w \leq -\textrm{mid} \mathbf{b} \\
		Ax - \textrm{diag}(\textrm{rad} \mathbf{b}) w \leq \textrm{mid} \mathbf{b}
	\end{cases}
	\Longleftrightarrow 
	\begin{cases}
	\textrm{mid} \mathbf{b} - \textrm{diag}(\textrm{rad} \mathbf{b}) w \leq Ax\\
	Ax \leq \textrm{mid} \mathbf{b} + \textrm{diag}(\textrm{rad} \mathbf{b}) w
	\end{cases}
		\Longleftrightarrow 
	\end{equation*}
	\begin{equation}
	\inf \tilde{\mathbf{b}} \leq Ax \leq \sup \tilde{\mathbf{b}}
	\end{equation}
\end{remark}

Данная задача ЛП успешно решается симплекс-методом. Программной реализацией на языке Python служит метод \texttt{linprog} пакета \texttt{scipy.optimize}.