В результате проделанной работы была скорректирована правая часть ИСЛАУ с точечной матрицей. Найденное решение в самом деле обеспечило непустоту доускового множества решения, а значит, сработало корректно.

В то же время стоит отметить, что радиус интервала был увеличен очень значительно: с \texttt{0.100} до \texttt{4.867}, что зачастую может быть неприемлемо в реальных задачах.