В результате проделанной работы была скорректирована правая часть ИСЛАУ с точечной матрицей. Найденное решение в самом деле обеспечило непустоту доускового множества решения, а значит, сработало корректно.

В то же время стоит отметить, что радиус интервала был увеличен очень значительно: с \texttt{0.100} до \texttt{0.467}, что зачастую может быть неприемлемо в реальных задачах.

Произведём простейшую оценку корректности полученного результата.
Радиусы правой части составляют:

\begin{equation}
\textrm{rad} \mathbf{b}=
\begin{pmatrix}
0.25 & 0.1 & 0.1
\end{pmatrix}^T
\end{equation}.

Значение максимума распознающего функционала в исходной системе равно приблизительно -0.2. Значит, используя жадный подход, можно расширить все радиусы на 0.2 и гарантированно достичь неотрицательности распознающего функционала и, следовательно, непустоты допускового множества решений. Таким образом, жадный вектор масштабных коэффициентов:

\begin{equation}
\mathbf{w_{greedy}}=
\begin{pmatrix}
\frac{0.25 + 0.2}{0.25} & \frac{0.1 + 0.2}{0.1} & \frac{0.1 + 0.2}{0.1}
\end{pmatrix}^T
=
\begin{pmatrix}
1.8 & 3 & 3
\end{pmatrix}^T
\end{equation}

и его норма: $\|\mathbf{w_{greedy}}\|_1=7.8$

Найденное симплекс-методом решение оптимальнее жадного практически в два раза (в смысле $l_1$-нормы).