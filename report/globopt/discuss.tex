Результаты решения первой задачи являются ожидаемыми: размер бруса уменьшается с каждой итерацией. Результаты решения второй задачи более интересны: видна тенденция к экспоненциальной сходимости.

Наблюдается следующее свойство: результат, который был достигнут на шестой итерации (первый ``большой выброс вниз''), становится стабильным (то есть алгоритм не имеет итераций с более поздним номером, доставляющих результат хуже данного) только приблизительно на сороковой итерации работы алгоритма, причём такие выбросы происходят регулярно (раз в двадцать итераций). 

После обнаружения этого свойства был построен такой же график для функции Растригина. Из полученного результата (см. \ref{pic:rastrigin_conv}) можно сделать вывод, что описанное свойство скорее связано со свойствами функции, нежели алгоритма, поскольку сходимость на функции Растригина имеет абсолютно иную форму: начиная с пятидесятой итерации метод не ухудшает результат.

Разрывы в графиках связаны с тем, что значение по оси ординат достигло области машинного нуля, поэтому не смогло быть корректно прологарифмировано.

На рисунках \ref{pic:rastrigin_traj} и \ref{pic:booth_traj} видно, что центры брусов не выходят за начальный диапазон значений, однако также хорошо видно, особенно у функции Бута, что сходиться к точке минимума центры начинают не сразу, а начиная с некоторой итерации, что ещё раз подтверждает выводы, сделанные при анализе графиков сходимости.