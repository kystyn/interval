Для демонстрации интервальной глобальной оптимизации требуется использовать функцию \cite{globopt}:
\begin{equation}
	function [Z, WorkList] = globopt0(X)
\end{equation}

Данная функция возвращает точку глобального экстремума $Z$ и рабочий список $WorkList$.

\subsection{Задача 1}
Рассмотреть пример из лекционного материала (функцию Растригина). Построить рабочий список и график сужения интервала.

\subsection{Задача 2}
Взять пример с сайта \cite{optfunc}. Изучить сходимость метода.